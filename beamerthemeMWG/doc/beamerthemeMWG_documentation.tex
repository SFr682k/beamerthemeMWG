%%	This is file 'beamerthemeMWG_documentation.tex', Version 2017-12-07
%%	Copyright 2017 Sebastian Friedl <sfr682k@t-online.de>
%% 
%%	This work may be distributed and/or modified under the conditions of the LaTeX Project
%%	Public License, either version 1.3c of this license or (at your option) any later version.
%%	The latest version of this license is available at
%%		http://www.latex-project.org/lppl.txt
%%	and version 1.3c or later is part of all distributions of LaTeX version 2008-05-04 or later.
%%
%%	This work has the LPPL maintenace status 'maintained'.
%%	The current maintainer of this work is Sebastian Friedl.
%%
%%	This work consists of the files beamerthemeMWG.sty and beamerthemeMWG_documentation.tex
%%
%%	---------------------------------------------------------------------------------------------------------------------------------------------
%%
%%	The MWG beamer theme is considered as a beamertheme suitable for every possible use.
%%	It uses the red color and the logo of the Markgräfin Wilhelmine Gymnasium, Bayreuth.
%%
%%	I created the theme for personal use while being student there.
%%
%%	The MWG logo has been converted to a path from a *.png file with Inkscape and exported to TikZ code using the tikz-export library
%%	(https://github.com/kjellmf/svg2tikz).
%%	Due to the complexity of the logo, the TikZ source got quite huge; therefore, it takes a bit long to compile it.
%%	By default, the theme loads a simplified version of the logo which reduces the compiling time to one third.
%%
%%	---------------------------------------------------------------------------------------------------------------------------------------------
%%
%%	Please report bugs and other problems as well as suggestions for improvements to my email address (sfr682k@t-online.de).
%%
%%	---------------------------------------------------------------------------------------------------------------------------------------------



% !TeX spellcheck = en_US

% !TeX document-id = {b3b4668f-f5d8-4010-ac4e-2eb3098d15f4}
% !TeX TXS-program:compile=txs:///pdflatex/[--shell-escape]

\documentclass[11pt]{ltxdoc}
\usepackage[utf8]{inputenc}

\usepackage[english]{babel}

\usepackage{amsmath,amssymb}
\usepackage{csquotes}
\usepackage{graphicx}
\usepackage{hologo}
\usepackage{hyperref}
\usepackage{minted}
\usepackage{pgf}
\usepackage{ragged2e}
\usepackage{textcomp}
\usepackage{tikz}

\usepackage[charter]{mathdesign}
\usepackage[osf]{XCharter}
\usepackage[osf, scale=.92]{roboto}
\usepackage[defaultmono, scale=.9]{droidmono}

\parindent 0pt

% The color of the MWG logo
\definecolor{MWGRed}{RGB}{100,29,27}

\usepackage[left=4.50cm,right=2.75cm,top=3.25cm,bottom=2.75cm,nohead]{geometry}

\hyphenation{con-fi-gu-ra-tion ge-ne-ral ge-ne-ra-ted wri-ting}

\title{The MWG \LaTeX\ beamer theme}
\author{Sebastian Friedl \\ \href{mailto:sfr682k@t-online.de}{\texttt{sfr682k@t-online.de}}}
\date{December 7, 2017}

\hypersetup{pdftitle={The MWG \LaTeX\ beamer theme},pdfauthor={Sebastian Friedl}}

\begin{document}
	\maketitle
	\thispagestyle{empty}
	

	\begin{center} \itshape
		Dedicated to my teachers and fellow students in year 11 \\[1.25\smallskipamount]
		They showed me the beauty of \LaTeX \\
		as well as the several flaws of MS Office documents
	\end{center}
	
	\medskip
	\begin{abstract}
		\hspace{-1.5em}%
		The MWG beamer theme is considered as a beamertheme suitable for every possible use. It uses the red color and the logo of the Markgräfin Wilhelmine Gymnasium, Bayreuth.
	\end{abstract}
	

	\tableofcontents

	\clearpage
	
	\subsection*{Important note}
	\addcontentsline{toc}{subsection}{Important note}
	Since the MWG logo included in the footline by default consists of a Ti\textit{k}Z source up to 5\,850~lines long, presentations may compile very long time. This can be very nasty, especially during the process of creating the presentation. \\
	\emph{To reduce the amount of time required for compiling, the theme uses a simplified version of the logo.} Simplified means that details not visible to your audience got removed, resulting in a reduction of compilation time to approx. $\tfrac13$. However, the detailed version is still available when using the \texttt{hqlogo} option. \\
	You can remove the logo \emph{temporary} by passing the \texttt{draft} option or \emph{permanently} by passing other options. See section \ref{themeoptions} for further details.
	
	
	\subsection*{Dependencies and other requirements}
	\addcontentsline{toc}{subsection}{Dependencies and other requirements}
	The MWG theme requires \LaTeXe\ and -- in addition to the ones requested by the \texttt{beamer} class -- following packages:
	
	\medskip
	\DescribeMacro{appendixnumberbeamer}
	A simple solution for appendix frames not being calculated into the total number of frames
	
	\medskip
	\DescribeMacro{etoolbox}
	Provides access on \hologo{eTeX} primitives
	
    \medskip
	\DescribeMacro{tikz}
    The frontend to \texttt{pgf} used for drawing background and logo
	
	
	\subsection*{Call for cooperation}
	\addcontentsline{toc}{subsection}{Call for cooperation}
	Please report bugs and other problems as well as suggestions for improvements to my email address (\href{mailto:sfr682k@t-online.de}{\texttt{sfr682k@t-online.de}}).
	
	
	\subsection*{Style sample}
	\addcontentsline{toc}{subsection}{Style sample}
	\begin{figure} \centering
		\pgfimage[width=0.465\textwidth,page=1]{sample-beamer-presentation_MWG}~~~\pgfimage[width=0.465\textwidth,page=2]{sample-beamer-presentation_MWG} \\[.5em]
		\pgfimage[width=0.465\textwidth,page=3]{sample-beamer-presentation_MWG}~~~\pgfimage[width=0.465\textwidth,page=4]{sample-beamer-presentation_MWG} \\[.5em]
		\pgfimage[width=0.465\textwidth,page=5]{sample-beamer-presentation_MWG}~~~\pgfimage[width=0.465\textwidth,page=6]{sample-beamer-presentation_MWG} \\[.5em]
		\pgfimage[width=0.465\textwidth,page=7]{sample-beamer-presentation_MWG}~~~\pgfimage[width=0.465\textwidth,page=8]{sample-beamer-presentation_MWG}
		
		\caption{Style sample of the MWG theme}
		\label{stylesample}
	\end{figure}
	
	The style sample shown in figure \ref{stylesample} was made using the sample presentation \enquote{Writing presentations in \LaTeX\ beamer?} created by Sebastian Friedl\footnote{Source available on \href{https://github.com/SFr682k/sample-latex-beamer-presentation}{GitHub} (\textit{WTFPL})}.
	
	
	\subsection*{License}
	\begin{small}
		\addcontentsline{toc}{subsection}{License}
		\textcopyright\ 2017 Sebastian Friedl
		
		\smallskip
		This work may be distributed and/or modified under the conditions of the \LaTeX\ Project Public License, either version 1.3c of this license or (at your option) any later version.
		
		\smallskip
		The latest version of this license is available at \url{http://www.latex-project.org/lppl.txt} and version 1.3c or later is part of all distributions of \LaTeX\ version 2008-05-04 or later.
	
		\smallskip
		This work has the LPPL maintenace status 'maintained'. The current maintainer of this work is Sebastian Friedl. \\
		This work consists of the following files:
		\begin{itemize} \itemsep 0pt
			\item \texttt{beamerthemeMWG.sty} and
			\item \texttt{beamerthemeMWG\_documentation.tex}
		\end{itemize}
	\end{small}
	
	\clearpage
	
	
	
	
	% BEGIN OF DOCUMENTATION PART
	
	\section{Using the theme}
	For using the theme you have to copy the file \texttt{beamerthemeMWG.sty} into the folder containing the master file of your presentation. Advanced users may also install the style file on their local system. \par
	After that, simply use the command \mintinline{LaTeX}{\usetheme{MWG}} to set the theme used in your presentation to the MWG theme.

	
	\section{Theme options}				\label{themeoptions}
	Passing some options to the theme influences the way it behaves. \\
	Syntax: \ \ \mintinline{LaTeX}{\usetheme[<option1>, <option2>, ...]{MWG}}
	
	\subsubsection*{Available options:}
	\DescribeMacro{nologo}
	No logos will be shown anywhere on the frame
		
	\medskip
	\DescribeMacro{draft}
	Prevents placement of the logo in the footline but keeps reserving the space. \\
	In contrast to the \texttt{draft} option of the \texttt{beamer} class, the other contents of the frame still stay the same and remain displayed.
    
    \medskip
    \DescribeMacro{externallogo}
    Removes the logo from the footline and the logo specified with the \mintinline{LaTeX}{\logo} command will be shown on the right--hand side directly above the footline
    
    \medskip
    \DescribeMacro{hqlogo}
    Uses the detailed version of the logo instead of the simplified one
    
    \medskip
	\DescribeMacro{nosmallcaps}
    Apply this option if the used fonts don't provide a small caps shape
    
    \medskip
    \DescribeMacro{notoc}
    This option prevents the navigation being placed in the headline, resulting in an empty headline. Use the \texttt{noheadline} option for removing the complete headline.
    
    \medskip
    \DescribeMacro{noheadline}
    Removes the headline
    
    \medskip
	\DescribeMacro{smallfootline}
    Uses a footline half the size of the default footline


	\section{Features}
	There are some features allowing configuration and personalization of the MWG theme as well as easier writing the presentation's source.
	
	\subsection{Title graphic}
	The theme is capable of showing a graphic on the title-- and other structure frames. The title graphic used by the theme is declared with \mintinline{LaTeX}{\titlegraphic{<graphic>}}, where \texttt{<graphic>} represents a command like \mintinline{LaTeX}{\includegraphics} used for loading the title graphic itself.
	
	\smallskip
	\textit{Note:} \\
	The title and structure frames will have a slightly different layout when a title graphic is defined
	
	
	\subsection{Structure frames}
	When using the MWG theme there will be a separation frame generated when the \mintinline{LaTeX}{\appendix} command is set. \\
	In addition to that, other structure frames may be inserted -- this can happen either manually or automatically.
	
	\subsubsection*{Manual insertion of structure frames}
	\mintinline{LaTeX}{\partframe} -- a frame showing the current part \\
	\mintinline{LaTeX}{\sectionframe} -- a frame showing the current section \\
	\mintinline{LaTeX}{\subsectionframe} -- a frame showing the current section and subsection
	
	\medskip
	The commands can be used inside as well as outside a frame. \\
	If a command is used \textit{inside} a frame this frame will be used; please note that the elements of the structure frame may cover the other content placed in this frame. \\
	If a command is used \textit{outside} a frame the theme will generate one; this frame won't be calculated into the total number of frames and will have the same frame number as the following frame.
	
	
	\subsubsection*{Automatically insertion of structure frames}
	Commands activating automatically insertion: \\[\smallskipamount]\nopagebreak
	\begin{tabular}{rl}
		part frames & \mintinline{LaTeX}{\activatepartframes} \\
		section frames & \mintinline{LaTeX}{\activatesectionframes} \\
		subsection frames & \mintinline{LaTeX}{\activatesubsectionframes}
	\end{tabular}
	
	\bigskip
	Commands deactivating automatically insertion: \\[\smallskipamount]\nopagebreak
	\begin{tabular}{rl}
		part frames & \mintinline{LaTeX}{\deactivatepartframes} \\
		section frames & \mintinline{LaTeX}{\deactivatesectionframes} \\
		subsection frames & \mintinline{LaTeX}{\deactivatesubsectionframes}
	\end{tabular}
	
	\bigskip
	It is recommended to deactivate the automatically insertion of part frames before using the \mintinline{LaTex}{\appendix} command; otherwise there will be two separation frames generated.
	
	
	\section{Appropriate fonts}
	Many elements of the MWG theme use the \textsc{small caps} font shape. \\
	This can lead to some unwanted results \textit{(like sans--serif text mixed up with serif small caps)} when the default \LaTeX\ document font, Computer Modern is used. \\
	On the other hand, the theme does not require any font packages, since there may be some problems with engines like \hologo{XeLaTeX} or \hologo{LuaLaTeX}. \\
	Therefore, you should load some font packages on your own.
	
	\smallskip
	In following, recommended combinations are listed. \\
	For this documentation, the Charter \& \textsf{Roboto} combination is used.
	
	
	\subsection{Font combinations using \LaTeX\ packages}
	\raggedleft
	\paragraph*{Charter \& Roboto} \hfill
	\textit{supports:} \LaTeX, \hologo{pdfLaTeX}, $\sqrt{math}$ \nopagebreak\vspace{-.75em}
	\begin{minted}[gobble=2, tabsize=4]{LaTeX}
		\usepackage[charter]{mathdesign}
		\usepackage[osf]{XCharter}
		\usepackage[osf,scale=.92]{roboto}
		\renewcommand{\familydefault}{\sfdefault}
	\end{minted}
	
	\paragraph*{Charter \& Droid Sans} \hfill
	\textit{supports:} \LaTeX, \hologo{pdfLaTeX}, \hologo{XeLaTeX}, \hologo{LuaLaTeX}, $\sqrt{math}$ \\
	\textit{doesn't support:} \textsc{\textsf{sans--serif smallcaps}} \nopagebreak\vspace{-2em}
	\begin{minted}[gobble=2, tabsize=4]{LaTeX}
		\usepackage[charter]{mathdesign}
		\usepackage[scale=.85,defaultsans]{droidsans}
	\end{minted}
	
	\paragraph*{Utopia \& Source Sans Pro} \hfill
	\textit{supports:} \LaTeX, \hologo{pdfLaTeX}, $\sqrt{math}$ \\
	\textit{doesn't support:} \textsc{\textsf{sans--serif smallcaps}} \nopagebreak\vspace{-2em}
	\begin{minted}[gobble=2, tabsize=4]{LaTeX}
		\usepackage[utopia]{mathdesign}
		\usepackage[scale=.95]{sourcesanspro}	
	\end{minted}
	
	\paragraph*{Times \& Helvetica} \hfill
	\textit{supports:} \LaTeX, \hologo{pdfLaTeX}, \hologo{XeLaTeX}, \hologo{LuaLaTeX}, $\sqrt{math}$ \nopagebreak\vspace{-.75em}
	\begin{minted}[gobble=2, tabsize=4]{LaTeX}
		\usepackage[slantedGreek]{mathptmx}
		\usepackage[scaled=.92]{helvet}
	\end{minted}

	
	\justifying
	\subsection{Font combinations for \hologo{XeLaTeX} and \hologo{LuaLaTeX} using \texttt{fontspec}}
	Please check whether these fonts are installed on your local system before using this font combinations.
	The \texttt{fontspec} and \texttt{unicode-math} packages both require the document being compiled with \hologo{XeLaTeX} or \hologo{LuaLaTeX}.
	
	\raggedleft
	\paragraph*{Cambria, Calibri \& Consolas} \hfill
	\textit{supports:} \hologo{XeLaTeX}, \hologo{LuaLaTeX}, $\sqrt{math}$ \nopagebreak\vspace{-.75em}
	\begin{minted}[gobble=2, tabsize=4]{LaTeX}
		\usepackage{fontspec}
		\usepackage{unicode-math}
		\setmainfont{Cambria}
		\setmathfont{Cambria Math}
		\setsansfont[Scale=MatchLowercase]{Calibri}
		\setmonofont[Scale=MatchLowercase]{Consolas}
	\end{minted}
	\vspace{-.75em}
	Load the fonts with the \texttt{Numbers=OldStyle} option to obtain old style figures  \hfill \hspace{.01pt}
	
	\paragraph*{Constantia, Corbel \& Consolas} \hfill
	\textit{supports:} \hologo{XeLaTeX}, \hologo{LuaLaTeX} \nopagebreak\vspace{-.75em}
	\begin{minted}[gobble=2, tabsize=4]{LaTeX}
		\usepackage{fontspec}
		\setmainfont{Constantia}
		\setsansfont[Scale=MatchLowercase]{Corbel}
		\setmonofont[Numbers=OldStyle,Scale=MatchLowercase]{Consolas}
	\end{minted}
	
	
	\justifying
	\vfill
	\thispagestyle{empty}
	\listoffigures
\end{document}